\documentclass[prodmode,acmtoit]{acmsmall}

\usepackage[utf8]{inputenc}
\usepackage{multicol}

\newcommand{\code}[1]{{\tt #1}}
\newcommand{\prog}[1]{{\tt #1}}

\acmVolume{V}
\acmNumber{N}
\acmArticle{A}
\acmYear{2014}
\acmMonth{4}

\markboth{Casper B. Hansen (fvx507)}{Individual Assignment 1}

\title{Datanet | Individual Assignment 1}

\author{CASPER B. HANSEN (FVX507) \affil{University of Copenhagen}}

\begin{abstract}
    An overview of the basic tools used for analyzing the behaviour of network
    systems. In a practical approach by experimenting with these tools we lead
    into a discussion of more theoretical topics in network technologies.
\end{abstract}

% \category{}{}{}

\terms{Experimentation, Measurement}
\keywords{Network, Tools}
\acmformat{Casper B. Hansen. 2014. Datanet, Individual Assignment 1.}

\begin{document}

\maketitle
% \tableofcontents

\section{Introduction}
\label{sec:introduction}
I will briefly go over the system and network setup used to perform the
practical aspects discussed throughout the document. Having provided these one
can reason the results of the measurements to be discussed later.

\begin{multicols}{2}

    \subsection{System Setup}
    \label{sec:introduction|sub:system-setup}
    The system used ran in the virtual machine environment VirtualBox (4.3.8).
    \\\\
    \begin{tabular}{ll}
        {\bf OS / Kernel}   & Arch Linux \\
        {\bf }              & 3.14.1-1-ARCH \\
        {\bf Shell}         & /bin/zsh \\
        {\bf User}          & casperbhansen \\
        {\bf Hostname}      & arch \\
    \end{tabular}
    \\\\
    All programs used were acquired through {\it pacman}, which is the
    standard package manager for the Arch Linux distribution.
    
    \columnbreak

    \subsection{Network Setup}
    \label{sec:introduction|sub:network-setup}
    The network was driven via the host computer, as a bridged connection,
    through the host system running Mac OS 10.9.2.
    \\\\
    \begin{tabular}{ll}
        {\bf Speed (Up/Down)}   & 2 Mbit / 20 Mbit \\
        {\bf Connection}        & Wireless \\
    \end{tabular}
    \\\\
    Because the connection is bridged, the test results should suffer little
    impact. However, since the connection is wireless some loss of speed is
    to be expected.

\end{multicols}

\section{Latency and Bandwidth}
\label{sec:latency-and-bandwidth}
By experimentation with tools like \prog{ping}, \prog{traceroute} and
\prog{wget} on the chosen target network hosts and a discussion of the
results we can attempt to draw a few conclusions.

\subsection{Chosen Targets}
\label{sec:latency-and-bandwidth|sub:chosen-targets}
For convenience, I have listed the target network hosts that will be tested
against. Each target is assigned a shorthand label for easy reference.
\begin{center}
    \begin{tabular}{|c|c|c|}
        \hline
        {\bf Label}     & {\bf Location}    & {\bf URL} \\ \hline
        AU  & Australia & {\tt http://ftp.au.debian.org/debian/} \\ \hline
        JP  & Japan     & {\tt http://ftp.jp.debian.org/debian/} \\ \hline
        DK  & Denmark   & {\tt http://ftp.dk.debian.org/debian/} \\ \hline
        UK  & England   & {\tt http://ftp.uk.debian.org/debian/} \\ \hline
        US  & America   & {\tt http://ftp.us.debian.org/debian/} \\ \hline
    \end{tabular}
\end{center}
The targets were chosen with the intend of producing a variety of results. The
tests were conducted in Denmark, making the Danish mirror the closest. England
comes in second as it is relatively close to Denmark. For the remainder of the
targets many factors can influence which is the fastest and slowest, so I
won't be making any guesses --- these were selected for exactly this reason.

\subsection{Ping}
The \prog{ping} program sends out an ICMP\footnote{ICMP --- Acronym denoting
{\it Internet Control Message Protocol}.} \code{ECHO\_REQUEST} datagram,
consisting of an IP-address  ICMP header and a timeval structure[\prog{ping}
manual page]. An \code{ECHO\_REQUEST} can be thought of as the {\it ``ping''}
itself, which basically asks the target network host to send back an ICMP
\code{ECHO\_REPLY}, also known as a {\it ``pong''}.

\begin{figure}[H]
    \center
    \begin{tabular}{|c|c|c|c|}
        \hline
        {\bf Target} & {\bf Minimum} & {\bf Average} & {\bf Maximum} \\ \hline
        DK & 31.649 ms & 33.972 ms & 36.685 ms \\ \hline
        UK & 42.757 ms & 52.393 ms & 71.944 ms \\ \hline
        US & 120.464 ms & 193.454 ms & 243.455 ms \\ \hline
        JP & 350.175 ms & 391.000 ms & 434.614 ms \\ \hline
        AU & 424.835 ms & 461.892 ms & 502.756 ms \\ \hline
    \end{tabular}
    \caption{Ping results}
    \label{table:ping}
\end{figure}

The above table

\section{HTTP Protocol}
\label{sec:http-protocol}
...

% Single appendix
%
% \section*{APPENDIX}
% \setcounter{section}{1}
% ...

% Multiple appendices
%
% \appendix
% \section*{APPENDIX}
% \section{First}
% \section{Second}

% \ack{}

\bibliographystyle{ACM-Reference-Format-Journals}

\begin{bottomstuff}
% additional bottom stuff
\end{bottomstuff}

\end{document}

